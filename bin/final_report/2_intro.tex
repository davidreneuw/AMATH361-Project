\chapter{Introduction}

A real world application of continuum mechanics is the study of how different elastic waves propagate in the Earth's crust. This field, called Seismology, commonly thought of as a very much applied science, stands on a large theoretical base. With applications ranging from detection of ground composition and structure\cite{https://doi.org/10.1029/95JB00259} to providing Earthquake early warnings\cite{government_of_canada_2019}, our group found interest in delving deeper into this theoretical base.

As part of venturing into this field, we identified a problem that seemed solvable and attainable within the scope of this project. The first step when taking a set of equations and making a physical model out of it is to evaluate the different effects caused by the real world onto our equations. In our case, we put the focus on what the waves propagate in during seismic events: the soil. Typically, due to the large scale of seismic events, surface elastic waves will encounter different types of soils and these soils will have considerable effects on the propagation and amplitude of waves.

